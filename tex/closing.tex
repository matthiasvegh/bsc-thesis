\chapter{Closing}
\label{chap:Closing}

\section{Results}
Several programs were implemented, that comprise a tool for suggesting
variable-names for \CC{} codebases. This tool was tested automatically using
Robot \cite{robotframework}, and tested manually against Xerces-\CC{}
\cite{xerces-c}. Variable-name suggestions were based on the type of the
variable in question, and suggestions and renaming was made available via a Vim
plugin.

\section{Further Work}
Multiple improvements can be made to the way variable-name suggestions work,
these are listed here.

\subsection{Variable-kind Aware Suggestions}
Several projects decorate variable names according to their kind, such as
\lstinline|m_| being used as a prefix for member variables. In Xerces-\CC{}, the
prefix  \lstinline|f| is used.

This should be incorporated into to the suggestion process, so that when
\lstinline|suggest_names| finds a candidate variable of kind $A$, for variable
of kind $B$, the prefix (or other decoration) for $A$ is removed, and the prefix
for $B$ is applied, thereby ensuring that variable-names suggested follow these
conventions.

\subsection{Only Suggest Names that can be Used}
\Fref{sec:troubleshooting} describes some error-cases, one of which being that
\lstinline|suggest_names| can suggest variable names after which the code fails
to compile. A solution to this would be to track a list of conflicting variable
names for each variable in the database. A variable $A$ would conflict with a
variable $B$, that is $B$ could not be renamed to $A$, if $A$ was declared in
the same scope as $B$ (as opposed to an ancestor scope), or if $A$ was declared
in an ancestor scope, but was referred to in the scope of $B$. This analysis
would need to be performed by the Clang-plugin, as determining conflicts like
this is \CC{} specific.

\subsection{Follow Function-calls}
Currently, suggestions are made purely based on the type of the variable. For
certain types, the type does not strongly determine the name of the variable.
For instance, a variable of type \lstinline|int| could legitmately be called
\lstinline|i|, \lstinline|fd|, \lstinline|error|, or
\lstinline|temperatureOfBudapestInCelsius|. While many types can reasonably
determine the names, types like \lstinline|int| do not. In this case, we can
follow the functions into which a variable is passed, and see what name was
assigned to the parameter in the definition of the function, as well as the name
of the returned value (if any). The current architecture of having several
databases and then merging them would work well for this, as the function
definition wouldn't need to be in the translation-unit where the rename takes
place.

\subsection{Ignore Certain Paths}
Many code-bases have dependencies on large third-party code-bases, such as the
\CC{} standard library, or Boost. Frequently however, code-bases use different
coding conventions to that of the third-party library, so those could be omitted
both for improving the results of variable-name suggestions. If they were
omitted during the database-generation phase, merging would even be faster as
there would be less variables to parse. For instance, a simple
\lstinline|#include<iostream>| typically generates upwards of 4500 variable
declarations, many of which are internal to the library and are of no use in
renaming.

\subsubsection{Map conventions}
A more sophisticated version of the above, is to distinguish parts of the
code-base that are written with different conventions. For instance, if a
code-base calls file descriptors \lstinline|fileDescriptor|, whereas the GNU C
Library \cite{glibc} calls them \lstinline|__fd|, instead of ignoring calls into
glibc, we could determine the convention as it applies in glibc, and map it back
to the calling project's conventions.

\subsection{Speed}
Currently, there is a non-trivial performance cost of running variable-name
analysis during the compilation. For instance, on an Intel Core i5-4590 CPU,
Xerces-\CC{} without our plugin builds in ~17s, where as takes upwards of 5
minutes with the plugin enabled. The primary cause of this is needing to track
whether we have seen the variables declaration before adding an occurence. This
is currently done with an array of Variables, storing an array of Occurences,
which leads to cubic time-complexity. Using a hashmap or similar data structure
would cut down the performance penalty drastically.
