\chapter{Introduction}
\label{chap:Introduction}

\epigraph{There are only two hard things in Computer Science: cache invalidation
	and naming things.}{Phil Karlton}

Several programs were written as part of this thesis, which when used together,
assist a software developer in giving variables in their \CC{} code-base good
names.

The primary goal of this thesis is to provide a refactoring tool that is aware
of all variable declarations and their relations, so that accurate suggestions
can be made. This is achieved by providing a compiler-plugin for Clang, the
\CC{}-compiler of the LLVM compiler infrastructure \cite{llvm}.

The compiler-plugin generates a database containing information on variables,
which are then used by a set of scripts to suggest variable names, and to
perform renaming operations.

A Vim-plugin is also provided to allow Vim-users to integrate the tools provided
into their daily workflow.

\section{Thesis Structure}
This thesis is composed of multiple chapters, each dealing with a certain part
of the work involved.
\Fref{chap:Introduction} provides the motivation for this project.
\Fref{chap:UserDoc} provides a guide on how to build, install and use the
project.
\Fref{chap:DevDoc} has detailed descriptions of the implementation,
algorithms used, and the architectural overview.
\Fref{chap:Testing} describes the testing, including a list of tests written.
The last \fref{chap:Summary} summarizes the work, and what further enhancements
can be made.
