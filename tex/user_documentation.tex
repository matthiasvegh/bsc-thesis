\chapter{User Documentation}
\label{chapter:UserDoc}

\newread\ownread
\newcommand{\piprequirements}[1]{
	\IfFileExists{#1}{
		\openin\ownread=#1
		\begin{inparaitem}
			\loop\unless\ifeof\ownread
			\read\ownread to \requirement
			\ifeof\ownread
			\else
				\item \requirement \cite{\requirement}
			\fi
			\repeat
		\end{inparaitem}
		\closein\ownread
	}{File not found}
}

\newcommand{\requirementsinstall}[1]{
	\IfFileExists{#1}{%
		\openin\ownread=#1
		\loop\unless\ifeof\ownread
		\read\ownread to \requirement
		\ifeof\ownread
		\else
			\requirement\ 
		\fi
		\repeat
		\closein\ownread
	}{File not found}
}

This chapter describes how to use the various components included in this
thesis.

\section{Building and Installation}
\subsection{Prerequisites}
\subsubsection{Python packages required}
To build, install and test the components of this project, the following
dependencies are required:
\piprequirements{requirements.txt}

These can be installed with:
\begin{lstlisting}[caption={Installing pip requirements}, escapechar=-]
pip install-\requirementsinstall{requirements.txt}-
\end{lstlisting}

\subsubsection{Apt packages required}
\piprequirements{apt_packages.txt}

These can be installed with:
\begin{lstlisting}[caption={Installing apt requirements}, escapechar=>,
		breaklines=true]
apt-get install>\requirementsinstall{apt_packages.txt}>
\end{lstlisting}

\subsubsection{\LaTeX{} packages required}
\begin{itemize}
	\item Tikz-UML \cite{tikzuml}
\end{itemize}

\subsection{Building}
Once the repository is cloned, setup a build directory and invoke meson.
\begin{lstlisting}[language=bash]
cd /path/to/repository
mkdir build
cd build
meson ..
cd ..
\end{lstlisting}

The default build target will build the project, along with this thesis.
\begin{lstlisting}[language=bash]
ninja -C build
\end{lstlisting}

To run the tests, use \lstinline|ninja -C build test|.

\subsection{Installing}
Add \lstinline|suggest_names| to your \lstinline|PATH|.
\begin{lstlisting}[language=bash]
$ export PATH=/path/to/repository/suggest_names:$PATH
\end{lstlisting}


\section{Using}
\subsection{Generating the Variable Database}
Before any variable suggestions can be made, the source-code must be indexed.
This is achieved with a clang-plugin, that collects all variable occurences,
and information about them, such as type and name.

\subsubsection{Clang Invocation}
To generate a database for a given source file, run clang as follows:

\begin{lstlisting}[caption={Invocation of Clang}]
clang++ \
	-fplugin=/path/to/repository/build/libdump_names.so \
	-Xclang \
	-plugin-arg-dump-names \
	-Xclang \
	output-file \
	-Xclang \
	-plugin-arg-dump-names \
	-Xclang \
	/path/for/database \
	all... \
	other... \
	args... \
	/path/to/source.cpp
\end{lstlisting}

If \lstinline|output-file| is not specified, the output file is derived from the
output of compilation. For instance:

\begin{lstlisting}[caption={Invocation of Clang with Defaulted Output Filename}]
clang++ \
	-fplugin=/path/to/repository/build/libdump_names.so \
	-c
	/path/to/source.cpp
	-o /path/to/output.o
\end{lstlisting}

Results in \lstinline|/path/to/output/o.names.json|. This is most useful when
using the Clang-plugin as part of a build-system, as described next.

\subsubsection{Build-System Integration}
To generate the variable databases for a project larger than a few files,
integration into the project's build-system is needed. As variable databases are
generated per translation-unit, this means we have to change to way \CC{} files
are compiled, namely that Clang is used as the compiler, and that our plugin is
used during compilation.

Several build-systems provide the option of the user to specify the compiler to
be used, as well as additional compiler arguments. For CMake \cite{cmake}, these
can be achieved as follows:

Set the compiler to Clang with:
\begin{lstlisting}[caption={CMake Compiler Setting}]
CMAKE_CXX_COMPILER=clang++
\end{lstlisting}

Set the compiler flags to use the plugin with:
\begin{lstlisting}[caption={CMake Compiler Arguments}]
CMAKE_CXX_FLAGS='-fplugin=/path/to/libdump_names.so'
\end{lstlisting}

For instance, when bootstrapping a project, instead of
\begin{lstlisting}[caption={CMake Invocation without the Plugin}, language=bash]
mkdir build
cd build
cmake -G "Ninja" ..
\end{lstlisting}
do
\begin{lstlisting}[caption={CMake Invocation with the Plugin}, language=bash]
mkdir build
cd build
cmake -G "Ninja" \
	-DCMAKE_CXX_COMPILER=clang++ \
	-DCMAKE_CXX_FLAGS='-fplugin=/path/to/libdump_names.so' \
	..
\end{lstlisting}

\subsubsection{An Example Run}
In the following sections we will refer to the following example piece of \CC{}
code:

\lstinputlisting[label={lst:examplecpp}, language=c++, numbers=left,
		caption={\lstinline|example.cpp|}]
		{example.cpp}

Running the Clang-plugin as described above will give us a JSON file containing
information on the symbols in the program:

\lstinputlisting[language=json, numbers=left,
		caption={\lstinline|examples.cpp.o.names.json|}]{example.varnames.json}

By default, all filenames in the output are full paths. To make the output more
easily legible, we pass the \lstinline|-fdebug-prefix-map=/path/to/repository=.|
command-line argument to Clang, so that the output files are relative the root
of the repository.

\subsection{Generating Variable-name Suggestions}
Once the code-base is indexed, suggestions can be requested with the script
\lstinline|suggest_names|.

\lstinline|suggest_names|' documentation is as follows:
\lstinputlisting[caption={\lstinline|suggest_names --help|}]
	{suggest_names_help.txt}

Invoking \lstinline|suggest_names| to suggest an alternative to
\lstinline|Foo::id| can be done as follows:

\begin{lstlisting}
suggest_names varnames.json ./examples/example.cpp 2 9
\end{lstlisting}

The resulting suggestions are as follows:
\lstinputlisting[caption={Suggestions for \lstinline|Foo::id|}]
		{example_suggestions}

\subsection{Performing a Rename}
Renames can be performed with \lstinline|suggest_names_rename|.
The documentation for \lstinline|suggest_names_rename| is as follows:
\lstinputlisting[caption={\lstinline|suggest_names_rename --help|}]
		{suggest_names_rename_help.txt}

To accept \lstinline|identifier| as the new name for \lstinline|Foo::id|, we can
then invoke \lstinline|suggest_names_rename| as follows:
\begin{lstlisting}[caption={Invocation of \lstinline|suggest_name_rename|}]
suggest_names_rename \
	--filename test.cpp --line 2 --column 6 \
	--name identifier \
	varnames.json
\end{lstlisting}

\subsection{Using the Vim Plugin}
To help making renaming easier, a Vim \cite{vim} plugin is included.
With it, the renaming process can be done interactively without needing to use
command-line tools.
The Vim plugin can be added to your \lstinline|.vimrc| by adding the following
two lines:

\begin{lstlisting}[language=vimscript, caption={Adding the Vim plugin}]
set rtp+=/path/to/repository/vim
helptags /path/to/repository/vim/doc
\end{lstlisting}

Following this, the documentation of the Vim plugin is available through the
usual commands of Vim, such as \lstinline|:help|.

To configure the path of variable databases, use:

\begin{lstlisting}[language=vimscript, caption={Setting Databases Path}]
let g:suggest_names_database_path = '/path/to/project/builddir'
\end{lstlisting}

The output of querying the help for \lstinline|:help suggest-names.txt| can be
found in appendix \ref{appendix:vimplugindoc}.

\section{Troubleshooting}
\label{sec:troubleshooting}
This section describes potential error messages the user may see during use of
the tools described prior, and how to resolve them.

\subsection{Variable not in merged database}
If the user requests variable-names for a variable not present in the database,
\lstinline|suggest_names| gives an error along the lines of:

\lstinputlisting[breaklines=true, caption=
		{Trackback when requesting names for a variable not present}]
		{example_failure}
In this case, the error was caused by querying a variable at line 2, column 10
of \lstinline|example.cpp| (see \fref{lst:examplecpp}). To fix this, we need to
pass column \emph{9} instead of 10 to get the proper result.

When using the Vim plugin, this does not occur even if the cursor is not on the
first character of the variable, as the Vim plugin first computes the position
of the start of the variable before requesting suggestions.

Also note, that column numbers are based on the number of characters that
precede them, not their apparent position when displayed on a screen. This is
mostly important in code-bases where tabs are used for indentation.

\subsection{Code Compilation Fails After a Rename}
If after a rename, the code no longer compiles, there are generally two possible
reasons \begin{inparaenum}[\itshape 1)\upshape]\item The variable-name chosen
was already in use and \item The rename was performed with out-of-date
databases\end{inparaenum}.
There is no general solution for the former, as this would require
\lstinline|suggest_names| to only suggest names that are not in use in the
surrounding scope, but the latter can be solved by always performing renames
right after build, where the databases are up to date.
